\selectlanguage{italian}%

\section{Sintesi su board FPGA}

\label{Sintesi ripple carry}

Per la sintesi si � utilizzato un tester simile a questo \ref{tester dispositivi seriali},
con la differenza che non vi � un process per la gestione dello start,
essendo il componente da testare combinatoriale, gli switch occorrono
per l' inserimento degli operandi i led sette , sei e cinque indicano,
nel caso in cui il led cinque o il led sei sia acceso che abbiamo
selezionato un registro per caricare i dati, se sono accesi entrambi
stiamo assegnando il valore uno al carry in ingresso, se si abilita
il led sette utilizziamo l' addizionatore come sottrattore, mentre
l' accensione del led zeo determina se � presente il carry in uscita
o meno, quella del led uno che vi � una situazione d' errore dovuta
al sottrattore, a dispetto del tester messo in riferimento, qui viene
anche utilizzato il display per visualizzare il risultato, per decidere
quante cifre del display vogliamo utilizzare basta abilitare gli ultimi
quattro switch (a partire dal quinto all' ottavo questi mettono in
funzione dalla prima alla quarta cifra) e premere il pulsante due,
se abbiamo bisogno dei punti questi vengono abilitati dai primi quattro
switch (stesso ragionamento fatto per le cifre) dopodich� bisogna
sempre premere il pulsante due, per visionare la somma bisogna premere
il pulsante uno.\selectlanguage{italian}%

