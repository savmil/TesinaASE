\selectlanguage{italian}%

\section{Traccia}

\subsection{Moltiplicatore a celle Mac}

Realizzare in VHDL un circuito di moltiplicazione a celle MAC di N
bit. La cella MAC deve contenere un Full Adder (descritto gi`a in
esercizi pre- cedenti) ed una porta AND per la moltiplicazione parziale.
Tale cella deve essere replicata in una struttura ordinata (per righe
e colonne) per comporre il circuito intero di moltiplicazione. Effettuare
considerazioni di occupazione di area e di tempi di propagazione dei
segnali al variare di N per valori significativi, apportando eventuali
commenti salienti.

\subsection{Moltiplicatore di Booth}

\noindent Realizzare in hardware l\textquoteright algoritmo della
moltiplica- zione secondo Booth per operandi ad 8 bit. L\textquoteright architettura
deve essere rea- lizzata sulla base dello schema di progettazione
PO/PC (Parte Operativa e Parte di Controllo).\selectlanguage{italian}%

