\selectlanguage{italian}%

\section{Traccia}

Progettare in VHDL un sommatore ad N bit capace di sommare 7 operandi.
Il risultato della somma deve essere completo senza overflow, cio`e
la somma in uscita deve essere espressa su un numero di bit tali da
consentire il non verificarsi di condizioni di overflow. Nel caso
specifico i bit in uscita devono essere pari ad N+3. Il progetto pu`o
essere realizzato mediante composizione di Full Adder. Per ciascuna
colonna si pu`o effettuare un conteggio, producendo in uscita un valore
binario espresso su 3 bit. Tenendo conto della posizione dei riporti
un sommatore (e.g. ripple carry) pu`o sommare tutti i riporti generati,
restituendo il risultato finale. I riporti generati possono propagarsi
sino alla cifra i+2 (e.g. la somma di 5 volte 1 genera 1 con riporto
10). Eventuali altre cifre binarie, come i riporti generati da somme
precedenti, vanno considerati: quindi si verifica che un riporto pu`o
generarsi fino alla i+4-esima cifra binaria.

\selectlanguage{italian}%

