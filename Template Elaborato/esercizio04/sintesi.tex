\selectlanguage{italian}%

\section{Sintesi su board FPGA\label{board disp}}

La soluzione adottata per interfacciare il design con la board fa
uso di una semplice unit� di controllo. Essa consente di ricevere
i valori di input tramite gli interruttori della board, e caricarli
nei registri relativi mediante i pulsanti. Il segnale di \emph{clock}
� associato al clock della board, il \emph{reset} ed i tre bit di
\emph{loader} sono associati ai quattro pulsanti, \emph{in\_byte }agli
otto interruttori, \emph{anodes} ai quattro anodi e \emph{cathodes}
agli otto catodi. Sono utilizzate tre batterie di otto flip flop D
edge triggered per salvare l'input nel giusto registro a seconda del
pulsante premuto. Esse, infatti, hanno per abilitazione ciascuna un
differente bit di caricamento (\emph{button)}, e quindi un differente
pulsante. Il reset negato ed i vari registri sono poi collegati al
display. Per i punti, il registro � collegato al display in forma
negata. Ci� perch� sono presi in input i punti da abilitare, abilitazione
che corrisponde a valle ad un abbassamento del catodo relativo. Infine,
si noti che gli anodi della board sono associati al valore negato
di quelli in uscita dal display: essi sono pilotati con una logica
0 attiva, di questo se ne occupa il componente display on board che
con un process controlla se � stato premuto il primo pulsante cos�
da usare l' input in ingresso per abilitare solo determinate cifre
o punti; con il secondo invece si carica il valore da voler visualizzare
sul display . 

\lstinputlisting[language=VHDL,caption={Architettura della Control Unit},firstline=32]{progetti/DCM/display_top_level.vhd}\selectlanguage{italian}%

