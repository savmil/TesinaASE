%% LyX 2.2.3 created this file.  For more info, see http://www.lyx.org/.
%% Do not edit unless you really know what you are doing.
\documentclass[12pt,oneside,american,english,italian]{book}
\renewcommand{\familydefault}{\rmdefault}
\usepackage[T1]{fontenc}
\usepackage[latin9]{luainputenc}
\usepackage{geometry}
\geometry{verbose,tmargin=2.5cm,bmargin=2cm,lmargin=2cm,rmargin=2cm}
\setcounter{secnumdepth}{3}
\setcounter{tocdepth}{3}
\usepackage{color}
\usepackage{array}
\usepackage{verbatim}
\usepackage{longtable}
\usepackage{float}
\usepackage{amssymb}
\usepackage{graphicx}

\makeatletter

%%%%%%%%%%%%%%%%%%%%%%%%%%%%%% LyX specific LaTeX commands.
%% Because html converters don't know tabularnewline
\providecommand{\tabularnewline}{\\}

%%%%%%%%%%%%%%%%%%%%%%%%%%%%%% User specified LaTeX commands.
% LOGO
\usepackage{eso-pic,graphicx}
\makeatletter
\newcommand\BackgroundPicture[2]{
\setlength{\unitlength}{1pt}
\put(0,\strip@pt\paperheight){
\parbox[t][\paperheight]{\paperwidth}{
\vfill
\centering\includegraphics[angle=#2]{#1}
\vfill
}
}
}
\makeatother
%Per i marigini
\sloppy

\usepackage{listings,xcolor,courier,bookmark}
\usepackage{listingsutf8}
\definecolor{darkblue}{named}{blue}
\definecolor{darkred}{named}{red}
\definecolor{grau}{named}{gray}
\let\Righttorque\relax
\lstset{
captionpos=b,
commentstyle=\color[rgb]{0.133,0.545,0.133},
keywordstyle=\color{darkblue},
stringstyle=\color{darkred},
extendedchars=true,
basicstyle=\small\ttfamily,
showstringspaces=false,
tabsize=2,
numbers=left,
numberstyle=\tiny,
breakautoindent  = true,
breakindent      = 2em,
breaklines       = true,
postbreak        = ,
prebreak         = \raisebox{-.8ex}[0ex][0ex]{\Righttorque},
showspaces=false, 
showtabs=false, 
showstringspaces=false,
language=VHDL,
frame=single,
morecomment=[s]{--}
}


\renewcommand*{\lstlistingname}{Codice Componente}

\usepackage{fancyhdr}
\pagestyle{fancy}

\fancyhead{} % cancella tutti i campi
\fancyfoot{} % cancella tutti i campi

\fancyhead[RO,LE]{\bfseries \leftmark}
\fancyfoot[LE,RO]{\thepage}
\fancyfoot[LO,CE]{Elaborato di ASE: Architettura dei Sistemi di Elaborazione}
\renewcommand{\headrulewidth}{0.4pt}
\renewcommand{\footrulewidth}{0.4pt}
\cfoot{}
\usepackage{tikz}
\usetikzlibrary{matrix,calc}

%isolated term
%#1 - Optional. Space between node and grouping line. Default=0
%#2 - node
%#3 - filling color
\newcommand{\implicantsol}[3][0]{
    \draw[rounded corners=3pt, fill=#3, opacity=0.3] ($(#2.north west)+(135:#1)$) rectangle ($(#2.south east)+(-45:#1)$);
    }


%internal group
%#1 - Optional. Space between node and grouping line. Default=0
%#2 - top left node
%#3 - bottom right node
%#4 - filling color
\newcommand{\implicant}[4][0]{
    \draw[rounded corners=3pt, fill=#4, opacity=0.3] ($(#2.north west)+(135:#1)$) rectangle ($(#3.south east)+(-45:#1)$);
    }

%group lateral borders
%#1 - Optional. Space between node and grouping line. Default=0
%#2 - top left node
%#3 - bottom right node
%#4 - filling color
\newcommand{\implicantcostats}[4][0]{
    \draw[rounded corners=3pt, fill=#4, opacity=0.3] ($(rf.east |- #2.north)+(90:#1)$)-| ($(#2.east)+(0:#1)$) |- ($(rf.east |- #3.south)+(-90:#1)$);
    \draw[rounded corners=3pt, fill=#4, opacity=0.3] ($(cf.west |- #2.north)+(90:#1)$) -| ($(#3.west)+(180:#1)$) |- ($(cf.west |- #3.south)+(-90:#1)$);
}

%group top-bottom borders
%#1 - Optional. Space between node and grouping line. Default=0
%#2 - top left node
%#3 - bottom right node
%#4 - filling color
\newcommand{\implicantdaltbaix}[4][0]{
    \draw[rounded corners=3pt, fill=#4, opacity=0.3] ($(cf.south -| #2.west)+(180:#1)$) |- ($(#2.south)+(-90:#1)$) -| ($(cf.south -| #3.east)+(0:#1)$);
    \draw[rounded corners=3pt, fill=#4, opacity=0.3] ($(rf.north -| #2.west)+(180:#1)$) |- ($(#3.north)+(90:#1)$) -| ($(rf.north -| #3.east)+(0:#1)$);
}

%group corners
%#1 - Optional. Space between node and grouping line. Default=0
%#2 - filling color
\newcommand{\implicantcantons}[2][0]{
    \draw[rounded corners=3pt, opacity=.3] ($(rf.east |- 0.south)+(-90:#1)$) -| ($(0.east |- cf.south)+(0:#1)$);
    \draw[rounded corners=3pt, opacity=.3] ($(rf.east |- 8.north)+(90:#1)$) -| ($(8.east |- rf.north)+(0:#1)$);
    \draw[rounded corners=3pt, opacity=.3] ($(cf.west |- 2.south)+(-90:#1)$) -| ($(2.west |- cf.south)+(180:#1)$);
    \draw[rounded corners=3pt, opacity=.3] ($(cf.west |- 10.north)+(90:#1)$) -| ($(10.west |- rf.north)+(180:#1)$);
    \fill[rounded corners=3pt, fill=#2, opacity=.3] ($(rf.east |- 0.south)+(-90:#1)$) -|  ($(0.east |- cf.south)+(0:#1)$) [sharp corners] ($(rf.east |- 0.south)+(-90:#1)$) |-  ($(0.east |- cf.south)+(0:#1)$) ;
    \fill[rounded corners=3pt, fill=#2, opacity=.3] ($(rf.east |- 8.north)+(90:#1)$) -| ($(8.east |- rf.north)+(0:#1)$) [sharp corners] ($(rf.east |- 8.north)+(90:#1)$) |- ($(8.east |- rf.north)+(0:#1)$) ;
    \fill[rounded corners=3pt, fill=#2, opacity=.3] ($(cf.west |- 2.south)+(-90:#1)$) -| ($(2.west |- cf.south)+(180:#1)$) [sharp corners]($(cf.west |- 2.south)+(-90:#1)$) |- ($(2.west |- cf.south)+(180:#1)$) ;
    \fill[rounded corners=3pt, fill=#2, opacity=.3] ($(cf.west |- 10.north)+(90:#1)$) -| ($(10.west |- rf.north)+(180:#1)$) [sharp corners] ($(cf.west |- 10.north)+(90:#1)$) |- ($(10.west |- rf.north)+(180:#1)$) ;
}

%Empty Karnaugh map 4x4
\newenvironment{Karnaugh}%
{
\begin{tikzpicture}[baseline=(current bounding box.north),scale=0.8]
\draw (0,0) grid (4,4);
\draw (0,4) -- node [pos=0.7,above right,anchor=south west] {cd} node [pos=0.7,below left,anchor=north east] {ab} ++(135:1);
%
\matrix (mapa) [matrix of nodes,
        column sep={0.8cm,between origins},
        row sep={0.8cm,between origins},
        every node/.style={minimum size=0.3mm},
        anchor=8.center,
        ampersand replacement=\&] at (0.5,0.5)
{
                       \& |(c00)| 00         \& |(c01)| 01         \& |(c11)| 11         \& |(c10)| 10         \& |(cf)| \phantom{00} \\
|(r00)| 00             \& |(0)|  \phantom{0} \& |(1)|  \phantom{0} \& |(3)|  \phantom{0} \& |(2)|  \phantom{0} \&                     \\
|(r01)| 01             \& |(4)|  \phantom{0} \& |(5)|  \phantom{0} \& |(7)|  \phantom{0} \& |(6)|  \phantom{0} \&                     \\
|(r11)| 11             \& |(12)| \phantom{0} \& |(13)| \phantom{0} \& |(15)| \phantom{0} \& |(14)| \phantom{0} \&                     \\
|(r10)| 10             \& |(8)|  \phantom{0} \& |(9)|  \phantom{0} \& |(11)| \phantom{0} \& |(10)| \phantom{0} \&                     \\
|(rf) | \phantom{00}   \&                    \&                    \&                    \&                    \&                     \\
};
}%
{
\end{tikzpicture}
}

%Empty Karnaugh map 2x4
\newenvironment{Karnaughvuit}%
{
\begin{tikzpicture}[baseline=(current bounding box.north),scale=0.8]
\draw (0,0) grid (4,2);
\draw (0,2) -- node [pos=0.7,above right,anchor=south west] {bc} node [pos=0.7,below left,anchor=north east] {a} ++(135:1);
%
\matrix (mapa) [matrix of nodes,
        column sep={0.8cm,between origins},
        row sep={0.8cm,between origins},
        every node/.style={minimum size=0.3mm},
        anchor=4.center,
        ampersand replacement=\&] at (0.5,0.5)
{
                      \& |(c00)| 00         \& |(c01)| 01         \& |(c11)| 11         \& |(c10)| 10         \& |(cf)| \phantom{00} \\
|(r00)| 0             \& |(0)|  \phantom{0} \& |(1)|  \phantom{0} \& |(3)|  \phantom{0} \& |(2)|  \phantom{0} \&                     \\
|(r01)| 1             \& |(4)|  \phantom{0} \& |(5)|  \phantom{0} \& |(7)|  \phantom{0} \& |(6)|  \phantom{0} \&                     \\
|(rf) | \phantom{00}  \&                    \&                    \&                    \&                    \&                     \\
};
}%
{
\end{tikzpicture}
}

%Empty Karnaugh map 2x2
\newenvironment{Karnaughquatre}%
{
\begin{tikzpicture}[baseline=(current bounding box.north),scale=0.8]
\draw (0,0) grid (2,2);
\draw (0,2) -- node [pos=0.7,above right,anchor=south west] {b} node [pos=0.7,below left,anchor=north east] {a} ++(135:1);
%
\matrix (mapa) [matrix of nodes,
        column sep={0.8cm,between origins},
        row sep={0.8cm,between origins},
        every node/.style={minimum size=0.3mm},
        anchor=2.center,
        ampersand replacement=\&] at (0.5,0.5)
{
          \& |(c00)| 0          \& |(c01)| 1  \\
|(r00)| 0 \& |(0)|  \phantom{0} \& |(1)|  \phantom{0} \\
|(r01)| 1 \& |(2)|  \phantom{0} \& |(3)|  \phantom{0} \\
};
}%
{
\end{tikzpicture}
}

%Defines 8 or 16 values (0,1,X)
\newcommand{\contingut}[1]{%
\foreach \x [count=\xi from 0]  in {#1}
     \path (\xi) node {\x};
}

%Places 1 in listed positions
\newcommand{\minterms}[1]{%
    \foreach \x in {#1}
        \path (\x) node {1};
}

%Places 0 in listed positions
\newcommand{\maxterms}[1]{%
    \foreach \x in {#1}
        \path (\x) node {0};
}

%Places X in listed positions
\newcommand{\indeterminats}[1]{%
    \foreach \x in {#1}
        \path (\x) node {X};
}
\hypersetup{hidelinks}
\usepackage[italian]{varioref}
\usepackage{caption}
\colorlet{BLACK}{black}
\captionsetup{tableposition=top,figureposition=bottom,font=small,format=hang,labelfont={sf,bf}}
\usepackage{hyperref}

\makeatother

\usepackage{babel}
\begin{document}
\begin{frontmatter}
\pagenumbering{Roman}

\title{Tesina di Architettura dei Sistemi di Elaborazione \\Gruppo 4} 

\author{Milo Saverio - Mat. M63/XXXX
\and Pommella Michele - Mat. M63/XXXX
\and Trimaldi Davide - Mat. M63/XXXX
}

\maketitle

\setcounter{page}{1}

\pagebreak{}

\tableofcontents{}

\pagebreak{}

\selectlanguage{american}%
\end{frontmatter}

\begin{mainmatter}

\selectlanguage{italian}%

\chapter{\textcolor{black}{Minimizzazione} reti combinatorie }

\selectlanguage{american}%

\AddToShipoutPicture{\BackgroundPicture{logo.png}{0}}

\selectlanguage{italian}%

\section{Traccia}

Realizzare un dispositivo VHDL che implementa il protocollo UART (a
partire da quello diffuso dalla Digilent). Collegare internamente,
oppure tramite inter- faccia fisica esterna alla board stessa, ad
un\textquoteright altra board oppure ad un PC previo utilizzo di un
physical RS232, due interfacce per trasmette e ricevere ottetti. Svolgere
l\textquoteright esercizio riutilizzando il VHDL messo a disposizione
da Digilent (e disponbile nel materiale del corso) commentando eventuali
ristrutturazioni del codice. (Opzionale) Sviluppare un\textquoteright architettura
per l\textquoteright implementazione del protocollo UART secondo il
paradigma PO/PC ex-novo, evidenziando le similarit`a/dissimilarit`a
con il progetto Digilent.\selectlanguage{italian}%



\selectlanguage{italian}%

\section{Soluzione}


\subsection{Schematico}

\subsubsection{GPIO}

\begin{figure}[H]
	\centering
	\includegraphics[scale=0.7]{esercizio16/images/GPIO.png}
	\caption{GPIO Schematic}
	\label{fig:GPIO_schematics}
\end{figure}

\subsection{Codice}

Progetto ISE: \href{run:progetti/GPIO/GPIO.xise}{GPIO ISE}


\subsubsection{Pad}

Questo componente, rappresentato in Figura \ref{fig:GPIO_schematics}
decide se l'operazione da effettuare � di scrittura o di lettura in
base al valore del segnale di enable. Quando enable � alto, si effettua
una scrittura e il segnale di input/output \textit{in\_out }viene
caricato con il valore del segnale di input, poi trasferito al segnale
di output. Quando enable � basso si effettua un'operazione di lettura
quindi, per evitare di disturbare il valore da leggere, viene spento
il buffer, ponendo il segnale in\_out ad alta impedenza ('Z'). Questo
comportamento � modellato tramite un costrutto dataflow \textit{with-select.}\\
Il componente in questione � osservabile a questo link: \href{run:progetti/GPIO/pad.vhd}{Pad}

\subsubsection{GPIO}

Questa � una top-level entity che si preoccupa di creare, tramite
un \textit{for-generate, }una serie di pad, il cui numero � settato
tramite un generic. \\
Il componente in questione � osservabile a questo link: \href{run:progetti/GPIO/gpio.vhd}{GPIO}\selectlanguage{italian}%



\selectlanguage{italian}%

\section{Simulazione}

In Figura \ref{fig:scan_chain_sim} � osservabile il funzionamento
del sistema appena descritto, utilizzando il testbench presente al
seguente link: \href{run:progetti/Boundary_Scan_Chain/boundary_scan_chain_testbench.vhd}{Boundary\_Scan\_Chain\_Testbench}.
\\
Come si nota � stata abilitata la modalit� shift register ponendo
ad 1 sia \textit{en}, che \textit{scan\_en}, in modo da shiftare il
valore 1 posto in ingresso a \textit{scan\_in}, che � visibile all'uscita
scan\_out esattamente dopo quattro colpi di clock, come c'era d'aspettarsi
dato che il numero di flip flop utilizzati � proprio quattro. Dopodich�
si � scelto di abilitare la modalit� registro e quindi a 55 ns sono
stati abbassati sia \textit{en} che \textit{scan\_en} e difatti all'uscita
\textit{dout} dei flip flop si osserva il valore 1010 posto in ingresso
precedentemente a \textit{din}.

\begin{figure}[H]
	\centering
	\includegraphics[scale=0.8]{esercizio06/images/scan_chain_simulazione.png}
	\caption{Simulazione della Boundary Scan Chain Behavioral}
	\label{fig:scan_chain_sim}
\end{figure}\selectlanguage{italian}%



\selectlanguage{italian}%

\section{Sintesi su board FPGA}

Fare riferimento a \ref{board disp}.\selectlanguage{italian}%

\selectlanguage{italian}%



\chapter{Reti combinatorie con l'ausilio di SIS e Mapping Tecnologico}

\selectlanguage{american}%

\AddToShipoutPicture{\BackgroundPicture{logo.png}{0}}

\selectlanguage{italian}%

\section{Traccia}

Realizzare un dispositivo VHDL che implementa il protocollo UART (a
partire da quello diffuso dalla Digilent). Collegare internamente,
oppure tramite inter- faccia fisica esterna alla board stessa, ad
un\textquoteright altra board oppure ad un PC previo utilizzo di un
physical RS232, due interfacce per trasmette e ricevere ottetti. Svolgere
l\textquoteright esercizio riutilizzando il VHDL messo a disposizione
da Digilent (e disponbile nel materiale del corso) commentando eventuali
ristrutturazioni del codice. (Opzionale) Sviluppare un\textquoteright architettura
per l\textquoteright implementazione del protocollo UART secondo il
paradigma PO/PC ex-novo, evidenziando le similarit`a/dissimilarit`a
con il progetto Digilent.\selectlanguage{italian}%



\selectlanguage{italian}%

\section{Soluzione}


\subsection{Schematico}

\subsubsection{GPIO}

\begin{figure}[H]
	\centering
	\includegraphics[scale=0.7]{esercizio16/images/GPIO.png}
	\caption{GPIO Schematic}
	\label{fig:GPIO_schematics}
\end{figure}

\subsection{Codice}

Progetto ISE: \href{run:progetti/GPIO/GPIO.xise}{GPIO ISE}


\subsubsection{Pad}

Questo componente, rappresentato in Figura \ref{fig:GPIO_schematics}
decide se l'operazione da effettuare � di scrittura o di lettura in
base al valore del segnale di enable. Quando enable � alto, si effettua
una scrittura e il segnale di input/output \textit{in\_out }viene
caricato con il valore del segnale di input, poi trasferito al segnale
di output. Quando enable � basso si effettua un'operazione di lettura
quindi, per evitare di disturbare il valore da leggere, viene spento
il buffer, ponendo il segnale in\_out ad alta impedenza ('Z'). Questo
comportamento � modellato tramite un costrutto dataflow \textit{with-select.}\\
Il componente in questione � osservabile a questo link: \href{run:progetti/GPIO/pad.vhd}{Pad}

\subsubsection{GPIO}

Questa � una top-level entity che si preoccupa di creare, tramite
un \textit{for-generate, }una serie di pad, il cui numero � settato
tramite un generic. \\
Il componente in questione � osservabile a questo link: \href{run:progetti/GPIO/gpio.vhd}{GPIO}\selectlanguage{italian}%



\selectlanguage{italian}%

\section{Simulazione}

In Figura \ref{fig:scan_chain_sim} � osservabile il funzionamento
del sistema appena descritto, utilizzando il testbench presente al
seguente link: \href{run:progetti/Boundary_Scan_Chain/boundary_scan_chain_testbench.vhd}{Boundary\_Scan\_Chain\_Testbench}.
\\
Come si nota � stata abilitata la modalit� shift register ponendo
ad 1 sia \textit{en}, che \textit{scan\_en}, in modo da shiftare il
valore 1 posto in ingresso a \textit{scan\_in}, che � visibile all'uscita
scan\_out esattamente dopo quattro colpi di clock, come c'era d'aspettarsi
dato che il numero di flip flop utilizzati � proprio quattro. Dopodich�
si � scelto di abilitare la modalit� registro e quindi a 55 ns sono
stati abbassati sia \textit{en} che \textit{scan\_en} e difatti all'uscita
\textit{dout} dei flip flop si osserva il valore 1010 posto in ingresso
precedentemente a \textit{din}.

\begin{figure}[H]
	\centering
	\includegraphics[scale=0.8]{esercizio06/images/scan_chain_simulazione.png}
	\caption{Simulazione della Boundary Scan Chain Behavioral}
	\label{fig:scan_chain_sim}
\end{figure}\selectlanguage{italian}%



\selectlanguage{italian}%

\section{Sintesi su board FPGA}

Fare riferimento a \ref{board disp}.\selectlanguage{italian}%

\selectlanguage{italian}%



\chapter{Latch/Flip Flop}

\selectlanguage{american}%

\AddToShipoutPicture{\BackgroundPicture{logo.png}{0}}

\selectlanguage{italian}%

\section{Traccia}

Realizzare un dispositivo VHDL che implementa il protocollo UART (a
partire da quello diffuso dalla Digilent). Collegare internamente,
oppure tramite inter- faccia fisica esterna alla board stessa, ad
un\textquoteright altra board oppure ad un PC previo utilizzo di un
physical RS232, due interfacce per trasmette e ricevere ottetti. Svolgere
l\textquoteright esercizio riutilizzando il VHDL messo a disposizione
da Digilent (e disponbile nel materiale del corso) commentando eventuali
ristrutturazioni del codice. (Opzionale) Sviluppare un\textquoteright architettura
per l\textquoteright implementazione del protocollo UART secondo il
paradigma PO/PC ex-novo, evidenziando le similarit`a/dissimilarit`a
con il progetto Digilent.\selectlanguage{italian}%



\selectlanguage{italian}%

\section{Soluzione}


\subsection{Schematico}

\subsubsection{GPIO}

\begin{figure}[H]
	\centering
	\includegraphics[scale=0.7]{esercizio16/images/GPIO.png}
	\caption{GPIO Schematic}
	\label{fig:GPIO_schematics}
\end{figure}

\subsection{Codice}

Progetto ISE: \href{run:progetti/GPIO/GPIO.xise}{GPIO ISE}


\subsubsection{Pad}

Questo componente, rappresentato in Figura \ref{fig:GPIO_schematics}
decide se l'operazione da effettuare � di scrittura o di lettura in
base al valore del segnale di enable. Quando enable � alto, si effettua
una scrittura e il segnale di input/output \textit{in\_out }viene
caricato con il valore del segnale di input, poi trasferito al segnale
di output. Quando enable � basso si effettua un'operazione di lettura
quindi, per evitare di disturbare il valore da leggere, viene spento
il buffer, ponendo il segnale in\_out ad alta impedenza ('Z'). Questo
comportamento � modellato tramite un costrutto dataflow \textit{with-select.}\\
Il componente in questione � osservabile a questo link: \href{run:progetti/GPIO/pad.vhd}{Pad}

\subsubsection{GPIO}

Questa � una top-level entity che si preoccupa di creare, tramite
un \textit{for-generate, }una serie di pad, il cui numero � settato
tramite un generic. \\
Il componente in questione � osservabile a questo link: \href{run:progetti/GPIO/gpio.vhd}{GPIO}\selectlanguage{italian}%



\selectlanguage{italian}%

\section{Simulazione}

In Figura \ref{fig:scan_chain_sim} � osservabile il funzionamento
del sistema appena descritto, utilizzando il testbench presente al
seguente link: \href{run:progetti/Boundary_Scan_Chain/boundary_scan_chain_testbench.vhd}{Boundary\_Scan\_Chain\_Testbench}.
\\
Come si nota � stata abilitata la modalit� shift register ponendo
ad 1 sia \textit{en}, che \textit{scan\_en}, in modo da shiftare il
valore 1 posto in ingresso a \textit{scan\_in}, che � visibile all'uscita
scan\_out esattamente dopo quattro colpi di clock, come c'era d'aspettarsi
dato che il numero di flip flop utilizzati � proprio quattro. Dopodich�
si � scelto di abilitare la modalit� registro e quindi a 55 ns sono
stati abbassati sia \textit{en} che \textit{scan\_en} e difatti all'uscita
\textit{dout} dei flip flop si osserva il valore 1010 posto in ingresso
precedentemente a \textit{din}.

\begin{figure}[H]
	\centering
	\includegraphics[scale=0.8]{esercizio06/images/scan_chain_simulazione.png}
	\caption{Simulazione della Boundary Scan Chain Behavioral}
	\label{fig:scan_chain_sim}
\end{figure}\selectlanguage{italian}%



\selectlanguage{italian}%

\section{Sintesi su board FPGA}

Fare riferimento a \ref{board disp}.\selectlanguage{italian}%

\selectlanguage{italian}%



\chapter{Display a 7 segmenti}

\selectlanguage{american}%

\AddToShipoutPicture{\BackgroundPicture{logo.png}{0}}

\selectlanguage{italian}%

\section{Traccia}

Realizzare un dispositivo VHDL che implementa il protocollo UART (a
partire da quello diffuso dalla Digilent). Collegare internamente,
oppure tramite inter- faccia fisica esterna alla board stessa, ad
un\textquoteright altra board oppure ad un PC previo utilizzo di un
physical RS232, due interfacce per trasmette e ricevere ottetti. Svolgere
l\textquoteright esercizio riutilizzando il VHDL messo a disposizione
da Digilent (e disponbile nel materiale del corso) commentando eventuali
ristrutturazioni del codice. (Opzionale) Sviluppare un\textquoteright architettura
per l\textquoteright implementazione del protocollo UART secondo il
paradigma PO/PC ex-novo, evidenziando le similarit`a/dissimilarit`a
con il progetto Digilent.\selectlanguage{italian}%



\selectlanguage{italian}%

\section{Soluzione}


\subsection{Schematico}

\subsubsection{GPIO}

\begin{figure}[H]
	\centering
	\includegraphics[scale=0.7]{esercizio16/images/GPIO.png}
	\caption{GPIO Schematic}
	\label{fig:GPIO_schematics}
\end{figure}

\subsection{Codice}

Progetto ISE: \href{run:progetti/GPIO/GPIO.xise}{GPIO ISE}


\subsubsection{Pad}

Questo componente, rappresentato in Figura \ref{fig:GPIO_schematics}
decide se l'operazione da effettuare � di scrittura o di lettura in
base al valore del segnale di enable. Quando enable � alto, si effettua
una scrittura e il segnale di input/output \textit{in\_out }viene
caricato con il valore del segnale di input, poi trasferito al segnale
di output. Quando enable � basso si effettua un'operazione di lettura
quindi, per evitare di disturbare il valore da leggere, viene spento
il buffer, ponendo il segnale in\_out ad alta impedenza ('Z'). Questo
comportamento � modellato tramite un costrutto dataflow \textit{with-select.}\\
Il componente in questione � osservabile a questo link: \href{run:progetti/GPIO/pad.vhd}{Pad}

\subsubsection{GPIO}

Questa � una top-level entity che si preoccupa di creare, tramite
un \textit{for-generate, }una serie di pad, il cui numero � settato
tramite un generic. \\
Il componente in questione � osservabile a questo link: \href{run:progetti/GPIO/gpio.vhd}{GPIO}\selectlanguage{italian}%



\selectlanguage{italian}%

\section{Simulazione}

In Figura \ref{fig:scan_chain_sim} � osservabile il funzionamento
del sistema appena descritto, utilizzando il testbench presente al
seguente link: \href{run:progetti/Boundary_Scan_Chain/boundary_scan_chain_testbench.vhd}{Boundary\_Scan\_Chain\_Testbench}.
\\
Come si nota � stata abilitata la modalit� shift register ponendo
ad 1 sia \textit{en}, che \textit{scan\_en}, in modo da shiftare il
valore 1 posto in ingresso a \textit{scan\_in}, che � visibile all'uscita
scan\_out esattamente dopo quattro colpi di clock, come c'era d'aspettarsi
dato che il numero di flip flop utilizzati � proprio quattro. Dopodich�
si � scelto di abilitare la modalit� registro e quindi a 55 ns sono
stati abbassati sia \textit{en} che \textit{scan\_en} e difatti all'uscita
\textit{dout} dei flip flop si osserva il valore 1010 posto in ingresso
precedentemente a \textit{din}.

\begin{figure}[H]
	\centering
	\includegraphics[scale=0.8]{esercizio06/images/scan_chain_simulazione.png}
	\caption{Simulazione della Boundary Scan Chain Behavioral}
	\label{fig:scan_chain_sim}
\end{figure}\selectlanguage{italian}%



\selectlanguage{italian}%

\section{Sintesi su board FPGA}

Fare riferimento a \ref{board disp}.\selectlanguage{italian}%

\selectlanguage{italian}%



\chapter{Clock Generator}

\selectlanguage{american}%

\AddToShipoutPicture{\BackgroundPicture{logo.png}{0}}

\selectlanguage{italian}%

\section{Traccia}

Realizzare un dispositivo VHDL che implementa il protocollo UART (a
partire da quello diffuso dalla Digilent). Collegare internamente,
oppure tramite inter- faccia fisica esterna alla board stessa, ad
un\textquoteright altra board oppure ad un PC previo utilizzo di un
physical RS232, due interfacce per trasmette e ricevere ottetti. Svolgere
l\textquoteright esercizio riutilizzando il VHDL messo a disposizione
da Digilent (e disponbile nel materiale del corso) commentando eventuali
ristrutturazioni del codice. (Opzionale) Sviluppare un\textquoteright architettura
per l\textquoteright implementazione del protocollo UART secondo il
paradigma PO/PC ex-novo, evidenziando le similarit`a/dissimilarit`a
con il progetto Digilent.\selectlanguage{italian}%



\selectlanguage{italian}%

\section{Soluzione}


\subsection{Schematico}

\subsubsection{GPIO}

\begin{figure}[H]
	\centering
	\includegraphics[scale=0.7]{esercizio16/images/GPIO.png}
	\caption{GPIO Schematic}
	\label{fig:GPIO_schematics}
\end{figure}

\subsection{Codice}

Progetto ISE: \href{run:progetti/GPIO/GPIO.xise}{GPIO ISE}


\subsubsection{Pad}

Questo componente, rappresentato in Figura \ref{fig:GPIO_schematics}
decide se l'operazione da effettuare � di scrittura o di lettura in
base al valore del segnale di enable. Quando enable � alto, si effettua
una scrittura e il segnale di input/output \textit{in\_out }viene
caricato con il valore del segnale di input, poi trasferito al segnale
di output. Quando enable � basso si effettua un'operazione di lettura
quindi, per evitare di disturbare il valore da leggere, viene spento
il buffer, ponendo il segnale in\_out ad alta impedenza ('Z'). Questo
comportamento � modellato tramite un costrutto dataflow \textit{with-select.}\\
Il componente in questione � osservabile a questo link: \href{run:progetti/GPIO/pad.vhd}{Pad}

\subsubsection{GPIO}

Questa � una top-level entity che si preoccupa di creare, tramite
un \textit{for-generate, }una serie di pad, il cui numero � settato
tramite un generic. \\
Il componente in questione � osservabile a questo link: \href{run:progetti/GPIO/gpio.vhd}{GPIO}\selectlanguage{italian}%



\selectlanguage{italian}%

\section{Simulazione}

In Figura \ref{fig:scan_chain_sim} � osservabile il funzionamento
del sistema appena descritto, utilizzando il testbench presente al
seguente link: \href{run:progetti/Boundary_Scan_Chain/boundary_scan_chain_testbench.vhd}{Boundary\_Scan\_Chain\_Testbench}.
\\
Come si nota � stata abilitata la modalit� shift register ponendo
ad 1 sia \textit{en}, che \textit{scan\_en}, in modo da shiftare il
valore 1 posto in ingresso a \textit{scan\_in}, che � visibile all'uscita
scan\_out esattamente dopo quattro colpi di clock, come c'era d'aspettarsi
dato che il numero di flip flop utilizzati � proprio quattro. Dopodich�
si � scelto di abilitare la modalit� registro e quindi a 55 ns sono
stati abbassati sia \textit{en} che \textit{scan\_en} e difatti all'uscita
\textit{dout} dei flip flop si osserva il valore 1010 posto in ingresso
precedentemente a \textit{din}.

\begin{figure}[H]
	\centering
	\includegraphics[scale=0.8]{esercizio06/images/scan_chain_simulazione.png}
	\caption{Simulazione della Boundary Scan Chain Behavioral}
	\label{fig:scan_chain_sim}
\end{figure}\selectlanguage{italian}%



\selectlanguage{italian}%

\section{Sintesi su board FPGA}

Fare riferimento a \ref{board disp}.\selectlanguage{italian}%

\selectlanguage{italian}%



\chapter{Scan Chain}

\selectlanguage{american}%

\AddToShipoutPicture{\BackgroundPicture{logo.png}{0}}

\selectlanguage{italian}%

\section{Traccia}

Realizzare un dispositivo VHDL che implementa il protocollo UART (a
partire da quello diffuso dalla Digilent). Collegare internamente,
oppure tramite inter- faccia fisica esterna alla board stessa, ad
un\textquoteright altra board oppure ad un PC previo utilizzo di un
physical RS232, due interfacce per trasmette e ricevere ottetti. Svolgere
l\textquoteright esercizio riutilizzando il VHDL messo a disposizione
da Digilent (e disponbile nel materiale del corso) commentando eventuali
ristrutturazioni del codice. (Opzionale) Sviluppare un\textquoteright architettura
per l\textquoteright implementazione del protocollo UART secondo il
paradigma PO/PC ex-novo, evidenziando le similarit`a/dissimilarit`a
con il progetto Digilent.\selectlanguage{italian}%



\selectlanguage{italian}%

\section{Soluzione}


\subsection{Schematico}

\subsubsection{GPIO}

\begin{figure}[H]
	\centering
	\includegraphics[scale=0.7]{esercizio16/images/GPIO.png}
	\caption{GPIO Schematic}
	\label{fig:GPIO_schematics}
\end{figure}

\subsection{Codice}

Progetto ISE: \href{run:progetti/GPIO/GPIO.xise}{GPIO ISE}


\subsubsection{Pad}

Questo componente, rappresentato in Figura \ref{fig:GPIO_schematics}
decide se l'operazione da effettuare � di scrittura o di lettura in
base al valore del segnale di enable. Quando enable � alto, si effettua
una scrittura e il segnale di input/output \textit{in\_out }viene
caricato con il valore del segnale di input, poi trasferito al segnale
di output. Quando enable � basso si effettua un'operazione di lettura
quindi, per evitare di disturbare il valore da leggere, viene spento
il buffer, ponendo il segnale in\_out ad alta impedenza ('Z'). Questo
comportamento � modellato tramite un costrutto dataflow \textit{with-select.}\\
Il componente in questione � osservabile a questo link: \href{run:progetti/GPIO/pad.vhd}{Pad}

\subsubsection{GPIO}

Questa � una top-level entity che si preoccupa di creare, tramite
un \textit{for-generate, }una serie di pad, il cui numero � settato
tramite un generic. \\
Il componente in questione � osservabile a questo link: \href{run:progetti/GPIO/gpio.vhd}{GPIO}\selectlanguage{italian}%



\selectlanguage{italian}%

\section{Simulazione}

In Figura \ref{fig:scan_chain_sim} � osservabile il funzionamento
del sistema appena descritto, utilizzando il testbench presente al
seguente link: \href{run:progetti/Boundary_Scan_Chain/boundary_scan_chain_testbench.vhd}{Boundary\_Scan\_Chain\_Testbench}.
\\
Come si nota � stata abilitata la modalit� shift register ponendo
ad 1 sia \textit{en}, che \textit{scan\_en}, in modo da shiftare il
valore 1 posto in ingresso a \textit{scan\_in}, che � visibile all'uscita
scan\_out esattamente dopo quattro colpi di clock, come c'era d'aspettarsi
dato che il numero di flip flop utilizzati � proprio quattro. Dopodich�
si � scelto di abilitare la modalit� registro e quindi a 55 ns sono
stati abbassati sia \textit{en} che \textit{scan\_en} e difatti all'uscita
\textit{dout} dei flip flop si osserva il valore 1010 posto in ingresso
precedentemente a \textit{din}.

\begin{figure}[H]
	\centering
	\includegraphics[scale=0.8]{esercizio06/images/scan_chain_simulazione.png}
	\caption{Simulazione della Boundary Scan Chain Behavioral}
	\label{fig:scan_chain_sim}
\end{figure}\selectlanguage{italian}%



\selectlanguage{italian}%

\section{Sintesi su board FPGA}

Fare riferimento a \ref{board disp}.\selectlanguage{italian}%

\selectlanguage{italian}%



\chapter{Finite State Machine}

\selectlanguage{american}%

\AddToShipoutPicture{\BackgroundPicture{logo.png}{0}}

\selectlanguage{italian}%

\section{Traccia}

Realizzare un dispositivo VHDL che implementa il protocollo UART (a
partire da quello diffuso dalla Digilent). Collegare internamente,
oppure tramite inter- faccia fisica esterna alla board stessa, ad
un\textquoteright altra board oppure ad un PC previo utilizzo di un
physical RS232, due interfacce per trasmette e ricevere ottetti. Svolgere
l\textquoteright esercizio riutilizzando il VHDL messo a disposizione
da Digilent (e disponbile nel materiale del corso) commentando eventuali
ristrutturazioni del codice. (Opzionale) Sviluppare un\textquoteright architettura
per l\textquoteright implementazione del protocollo UART secondo il
paradigma PO/PC ex-novo, evidenziando le similarit`a/dissimilarit`a
con il progetto Digilent.\selectlanguage{italian}%



\selectlanguage{italian}%

\section{Soluzione}


\subsection{Schematico}

\subsubsection{GPIO}

\begin{figure}[H]
	\centering
	\includegraphics[scale=0.7]{esercizio16/images/GPIO.png}
	\caption{GPIO Schematic}
	\label{fig:GPIO_schematics}
\end{figure}

\subsection{Codice}

Progetto ISE: \href{run:progetti/GPIO/GPIO.xise}{GPIO ISE}


\subsubsection{Pad}

Questo componente, rappresentato in Figura \ref{fig:GPIO_schematics}
decide se l'operazione da effettuare � di scrittura o di lettura in
base al valore del segnale di enable. Quando enable � alto, si effettua
una scrittura e il segnale di input/output \textit{in\_out }viene
caricato con il valore del segnale di input, poi trasferito al segnale
di output. Quando enable � basso si effettua un'operazione di lettura
quindi, per evitare di disturbare il valore da leggere, viene spento
il buffer, ponendo il segnale in\_out ad alta impedenza ('Z'). Questo
comportamento � modellato tramite un costrutto dataflow \textit{with-select.}\\
Il componente in questione � osservabile a questo link: \href{run:progetti/GPIO/pad.vhd}{Pad}

\subsubsection{GPIO}

Questa � una top-level entity che si preoccupa di creare, tramite
un \textit{for-generate, }una serie di pad, il cui numero � settato
tramite un generic. \\
Il componente in questione � osservabile a questo link: \href{run:progetti/GPIO/gpio.vhd}{GPIO}\selectlanguage{italian}%



\selectlanguage{italian}%

\section{Simulazione}

In Figura \ref{fig:scan_chain_sim} � osservabile il funzionamento
del sistema appena descritto, utilizzando il testbench presente al
seguente link: \href{run:progetti/Boundary_Scan_Chain/boundary_scan_chain_testbench.vhd}{Boundary\_Scan\_Chain\_Testbench}.
\\
Come si nota � stata abilitata la modalit� shift register ponendo
ad 1 sia \textit{en}, che \textit{scan\_en}, in modo da shiftare il
valore 1 posto in ingresso a \textit{scan\_in}, che � visibile all'uscita
scan\_out esattamente dopo quattro colpi di clock, come c'era d'aspettarsi
dato che il numero di flip flop utilizzati � proprio quattro. Dopodich�
si � scelto di abilitare la modalit� registro e quindi a 55 ns sono
stati abbassati sia \textit{en} che \textit{scan\_en} e difatti all'uscita
\textit{dout} dei flip flop si osserva il valore 1010 posto in ingresso
precedentemente a \textit{din}.

\begin{figure}[H]
	\centering
	\includegraphics[scale=0.8]{esercizio06/images/scan_chain_simulazione.png}
	\caption{Simulazione della Boundary Scan Chain Behavioral}
	\label{fig:scan_chain_sim}
\end{figure}\selectlanguage{italian}%



\selectlanguage{italian}%

\section{Sintesi su board FPGA}

Fare riferimento a \ref{board disp}.\selectlanguage{italian}%

\selectlanguage{italian}%



\chapter{Ripple Carry}

\selectlanguage{american}%

\AddToShipoutPicture{\BackgroundPicture{logo.png}{0}}

\selectlanguage{italian}%

\section{Traccia}

Realizzare un dispositivo VHDL che implementa il protocollo UART (a
partire da quello diffuso dalla Digilent). Collegare internamente,
oppure tramite inter- faccia fisica esterna alla board stessa, ad
un\textquoteright altra board oppure ad un PC previo utilizzo di un
physical RS232, due interfacce per trasmette e ricevere ottetti. Svolgere
l\textquoteright esercizio riutilizzando il VHDL messo a disposizione
da Digilent (e disponbile nel materiale del corso) commentando eventuali
ristrutturazioni del codice. (Opzionale) Sviluppare un\textquoteright architettura
per l\textquoteright implementazione del protocollo UART secondo il
paradigma PO/PC ex-novo, evidenziando le similarit`a/dissimilarit`a
con il progetto Digilent.\selectlanguage{italian}%



\selectlanguage{italian}%

\section{Soluzione}


\subsection{Schematico}

\subsubsection{GPIO}

\begin{figure}[H]
	\centering
	\includegraphics[scale=0.7]{esercizio16/images/GPIO.png}
	\caption{GPIO Schematic}
	\label{fig:GPIO_schematics}
\end{figure}

\subsection{Codice}

Progetto ISE: \href{run:progetti/GPIO/GPIO.xise}{GPIO ISE}


\subsubsection{Pad}

Questo componente, rappresentato in Figura \ref{fig:GPIO_schematics}
decide se l'operazione da effettuare � di scrittura o di lettura in
base al valore del segnale di enable. Quando enable � alto, si effettua
una scrittura e il segnale di input/output \textit{in\_out }viene
caricato con il valore del segnale di input, poi trasferito al segnale
di output. Quando enable � basso si effettua un'operazione di lettura
quindi, per evitare di disturbare il valore da leggere, viene spento
il buffer, ponendo il segnale in\_out ad alta impedenza ('Z'). Questo
comportamento � modellato tramite un costrutto dataflow \textit{with-select.}\\
Il componente in questione � osservabile a questo link: \href{run:progetti/GPIO/pad.vhd}{Pad}

\subsubsection{GPIO}

Questa � una top-level entity che si preoccupa di creare, tramite
un \textit{for-generate, }una serie di pad, il cui numero � settato
tramite un generic. \\
Il componente in questione � osservabile a questo link: \href{run:progetti/GPIO/gpio.vhd}{GPIO}\selectlanguage{italian}%



\selectlanguage{italian}%

\section{Simulazione}

In Figura \ref{fig:scan_chain_sim} � osservabile il funzionamento
del sistema appena descritto, utilizzando il testbench presente al
seguente link: \href{run:progetti/Boundary_Scan_Chain/boundary_scan_chain_testbench.vhd}{Boundary\_Scan\_Chain\_Testbench}.
\\
Come si nota � stata abilitata la modalit� shift register ponendo
ad 1 sia \textit{en}, che \textit{scan\_en}, in modo da shiftare il
valore 1 posto in ingresso a \textit{scan\_in}, che � visibile all'uscita
scan\_out esattamente dopo quattro colpi di clock, come c'era d'aspettarsi
dato che il numero di flip flop utilizzati � proprio quattro. Dopodich�
si � scelto di abilitare la modalit� registro e quindi a 55 ns sono
stati abbassati sia \textit{en} che \textit{scan\_en} e difatti all'uscita
\textit{dout} dei flip flop si osserva il valore 1010 posto in ingresso
precedentemente a \textit{din}.

\begin{figure}[H]
	\centering
	\includegraphics[scale=0.8]{esercizio06/images/scan_chain_simulazione.png}
	\caption{Simulazione della Boundary Scan Chain Behavioral}
	\label{fig:scan_chain_sim}
\end{figure}\selectlanguage{italian}%



\selectlanguage{italian}%

\section{Sintesi su board FPGA}

Fare riferimento a \ref{board disp}.\selectlanguage{italian}%

\selectlanguage{italian}%



\chapter{Carry Look Ahead}

\selectlanguage{american}%

\AddToShipoutPicture{\BackgroundPicture{logo.png}{0}}

\selectlanguage{italian}%

\section{Traccia}

Realizzare un dispositivo VHDL che implementa il protocollo UART (a
partire da quello diffuso dalla Digilent). Collegare internamente,
oppure tramite inter- faccia fisica esterna alla board stessa, ad
un\textquoteright altra board oppure ad un PC previo utilizzo di un
physical RS232, due interfacce per trasmette e ricevere ottetti. Svolgere
l\textquoteright esercizio riutilizzando il VHDL messo a disposizione
da Digilent (e disponbile nel materiale del corso) commentando eventuali
ristrutturazioni del codice. (Opzionale) Sviluppare un\textquoteright architettura
per l\textquoteright implementazione del protocollo UART secondo il
paradigma PO/PC ex-novo, evidenziando le similarit`a/dissimilarit`a
con il progetto Digilent.\selectlanguage{italian}%



\selectlanguage{italian}%

\section{Soluzione}


\subsection{Schematico}

\subsubsection{GPIO}

\begin{figure}[H]
	\centering
	\includegraphics[scale=0.7]{esercizio16/images/GPIO.png}
	\caption{GPIO Schematic}
	\label{fig:GPIO_schematics}
\end{figure}

\subsection{Codice}

Progetto ISE: \href{run:progetti/GPIO/GPIO.xise}{GPIO ISE}


\subsubsection{Pad}

Questo componente, rappresentato in Figura \ref{fig:GPIO_schematics}
decide se l'operazione da effettuare � di scrittura o di lettura in
base al valore del segnale di enable. Quando enable � alto, si effettua
una scrittura e il segnale di input/output \textit{in\_out }viene
caricato con il valore del segnale di input, poi trasferito al segnale
di output. Quando enable � basso si effettua un'operazione di lettura
quindi, per evitare di disturbare il valore da leggere, viene spento
il buffer, ponendo il segnale in\_out ad alta impedenza ('Z'). Questo
comportamento � modellato tramite un costrutto dataflow \textit{with-select.}\\
Il componente in questione � osservabile a questo link: \href{run:progetti/GPIO/pad.vhd}{Pad}

\subsubsection{GPIO}

Questa � una top-level entity che si preoccupa di creare, tramite
un \textit{for-generate, }una serie di pad, il cui numero � settato
tramite un generic. \\
Il componente in questione � osservabile a questo link: \href{run:progetti/GPIO/gpio.vhd}{GPIO}\selectlanguage{italian}%



\selectlanguage{italian}%

\section{Simulazione}

In Figura \ref{fig:scan_chain_sim} � osservabile il funzionamento
del sistema appena descritto, utilizzando il testbench presente al
seguente link: \href{run:progetti/Boundary_Scan_Chain/boundary_scan_chain_testbench.vhd}{Boundary\_Scan\_Chain\_Testbench}.
\\
Come si nota � stata abilitata la modalit� shift register ponendo
ad 1 sia \textit{en}, che \textit{scan\_en}, in modo da shiftare il
valore 1 posto in ingresso a \textit{scan\_in}, che � visibile all'uscita
scan\_out esattamente dopo quattro colpi di clock, come c'era d'aspettarsi
dato che il numero di flip flop utilizzati � proprio quattro. Dopodich�
si � scelto di abilitare la modalit� registro e quindi a 55 ns sono
stati abbassati sia \textit{en} che \textit{scan\_en} e difatti all'uscita
\textit{dout} dei flip flop si osserva il valore 1010 posto in ingresso
precedentemente a \textit{din}.

\begin{figure}[H]
	\centering
	\includegraphics[scale=0.8]{esercizio06/images/scan_chain_simulazione.png}
	\caption{Simulazione della Boundary Scan Chain Behavioral}
	\label{fig:scan_chain_sim}
\end{figure}\selectlanguage{italian}%



\selectlanguage{italian}%

\section{Sintesi su board FPGA}

Fare riferimento a \ref{board disp}.\selectlanguage{italian}%

\selectlanguage{italian}%



\chapter{Carry Save}

\selectlanguage{american}%

\AddToShipoutPicture{\BackgroundPicture{logo.png}{0}}

\selectlanguage{italian}%

\section{Traccia}

Realizzare un dispositivo VHDL che implementa il protocollo UART (a
partire da quello diffuso dalla Digilent). Collegare internamente,
oppure tramite inter- faccia fisica esterna alla board stessa, ad
un\textquoteright altra board oppure ad un PC previo utilizzo di un
physical RS232, due interfacce per trasmette e ricevere ottetti. Svolgere
l\textquoteright esercizio riutilizzando il VHDL messo a disposizione
da Digilent (e disponbile nel materiale del corso) commentando eventuali
ristrutturazioni del codice. (Opzionale) Sviluppare un\textquoteright architettura
per l\textquoteright implementazione del protocollo UART secondo il
paradigma PO/PC ex-novo, evidenziando le similarit`a/dissimilarit`a
con il progetto Digilent.\selectlanguage{italian}%



\selectlanguage{italian}%

\section{Soluzione}


\subsection{Schematico}

\subsubsection{GPIO}

\begin{figure}[H]
	\centering
	\includegraphics[scale=0.7]{esercizio16/images/GPIO.png}
	\caption{GPIO Schematic}
	\label{fig:GPIO_schematics}
\end{figure}

\subsection{Codice}

Progetto ISE: \href{run:progetti/GPIO/GPIO.xise}{GPIO ISE}


\subsubsection{Pad}

Questo componente, rappresentato in Figura \ref{fig:GPIO_schematics}
decide se l'operazione da effettuare � di scrittura o di lettura in
base al valore del segnale di enable. Quando enable � alto, si effettua
una scrittura e il segnale di input/output \textit{in\_out }viene
caricato con il valore del segnale di input, poi trasferito al segnale
di output. Quando enable � basso si effettua un'operazione di lettura
quindi, per evitare di disturbare il valore da leggere, viene spento
il buffer, ponendo il segnale in\_out ad alta impedenza ('Z'). Questo
comportamento � modellato tramite un costrutto dataflow \textit{with-select.}\\
Il componente in questione � osservabile a questo link: \href{run:progetti/GPIO/pad.vhd}{Pad}

\subsubsection{GPIO}

Questa � una top-level entity che si preoccupa di creare, tramite
un \textit{for-generate, }una serie di pad, il cui numero � settato
tramite un generic. \\
Il componente in questione � osservabile a questo link: \href{run:progetti/GPIO/gpio.vhd}{GPIO}\selectlanguage{italian}%



\selectlanguage{italian}%

\section{Simulazione}

In Figura \ref{fig:scan_chain_sim} � osservabile il funzionamento
del sistema appena descritto, utilizzando il testbench presente al
seguente link: \href{run:progetti/Boundary_Scan_Chain/boundary_scan_chain_testbench.vhd}{Boundary\_Scan\_Chain\_Testbench}.
\\
Come si nota � stata abilitata la modalit� shift register ponendo
ad 1 sia \textit{en}, che \textit{scan\_en}, in modo da shiftare il
valore 1 posto in ingresso a \textit{scan\_in}, che � visibile all'uscita
scan\_out esattamente dopo quattro colpi di clock, come c'era d'aspettarsi
dato che il numero di flip flop utilizzati � proprio quattro. Dopodich�
si � scelto di abilitare la modalit� registro e quindi a 55 ns sono
stati abbassati sia \textit{en} che \textit{scan\_en} e difatti all'uscita
\textit{dout} dei flip flop si osserva il valore 1010 posto in ingresso
precedentemente a \textit{din}.

\begin{figure}[H]
	\centering
	\includegraphics[scale=0.8]{esercizio06/images/scan_chain_simulazione.png}
	\caption{Simulazione della Boundary Scan Chain Behavioral}
	\label{fig:scan_chain_sim}
\end{figure}\selectlanguage{italian}%



\selectlanguage{italian}%

\section{Sintesi su board FPGA}

Fare riferimento a \ref{board disp}.\selectlanguage{italian}%

\selectlanguage{italian}%



\chapter{Carry Select}

\selectlanguage{american}%

\AddToShipoutPicture{\BackgroundPicture{logo.png}{0}}

\selectlanguage{italian}%

\section{Traccia}

Realizzare un dispositivo VHDL che implementa il protocollo UART (a
partire da quello diffuso dalla Digilent). Collegare internamente,
oppure tramite inter- faccia fisica esterna alla board stessa, ad
un\textquoteright altra board oppure ad un PC previo utilizzo di un
physical RS232, due interfacce per trasmette e ricevere ottetti. Svolgere
l\textquoteright esercizio riutilizzando il VHDL messo a disposizione
da Digilent (e disponbile nel materiale del corso) commentando eventuali
ristrutturazioni del codice. (Opzionale) Sviluppare un\textquoteright architettura
per l\textquoteright implementazione del protocollo UART secondo il
paradigma PO/PC ex-novo, evidenziando le similarit`a/dissimilarit`a
con il progetto Digilent.\selectlanguage{italian}%



\selectlanguage{italian}%

\section{Soluzione}


\subsection{Schematico}

\subsubsection{GPIO}

\begin{figure}[H]
	\centering
	\includegraphics[scale=0.7]{esercizio16/images/GPIO.png}
	\caption{GPIO Schematic}
	\label{fig:GPIO_schematics}
\end{figure}

\subsection{Codice}

Progetto ISE: \href{run:progetti/GPIO/GPIO.xise}{GPIO ISE}


\subsubsection{Pad}

Questo componente, rappresentato in Figura \ref{fig:GPIO_schematics}
decide se l'operazione da effettuare � di scrittura o di lettura in
base al valore del segnale di enable. Quando enable � alto, si effettua
una scrittura e il segnale di input/output \textit{in\_out }viene
caricato con il valore del segnale di input, poi trasferito al segnale
di output. Quando enable � basso si effettua un'operazione di lettura
quindi, per evitare di disturbare il valore da leggere, viene spento
il buffer, ponendo il segnale in\_out ad alta impedenza ('Z'). Questo
comportamento � modellato tramite un costrutto dataflow \textit{with-select.}\\
Il componente in questione � osservabile a questo link: \href{run:progetti/GPIO/pad.vhd}{Pad}

\subsubsection{GPIO}

Questa � una top-level entity che si preoccupa di creare, tramite
un \textit{for-generate, }una serie di pad, il cui numero � settato
tramite un generic. \\
Il componente in questione � osservabile a questo link: \href{run:progetti/GPIO/gpio.vhd}{GPIO}\selectlanguage{italian}%



\selectlanguage{italian}%

\section{Simulazione}

In Figura \ref{fig:scan_chain_sim} � osservabile il funzionamento
del sistema appena descritto, utilizzando il testbench presente al
seguente link: \href{run:progetti/Boundary_Scan_Chain/boundary_scan_chain_testbench.vhd}{Boundary\_Scan\_Chain\_Testbench}.
\\
Come si nota � stata abilitata la modalit� shift register ponendo
ad 1 sia \textit{en}, che \textit{scan\_en}, in modo da shiftare il
valore 1 posto in ingresso a \textit{scan\_in}, che � visibile all'uscita
scan\_out esattamente dopo quattro colpi di clock, come c'era d'aspettarsi
dato che il numero di flip flop utilizzati � proprio quattro. Dopodich�
si � scelto di abilitare la modalit� registro e quindi a 55 ns sono
stati abbassati sia \textit{en} che \textit{scan\_en} e difatti all'uscita
\textit{dout} dei flip flop si osserva il valore 1010 posto in ingresso
precedentemente a \textit{din}.

\begin{figure}[H]
	\centering
	\includegraphics[scale=0.8]{esercizio06/images/scan_chain_simulazione.png}
	\caption{Simulazione della Boundary Scan Chain Behavioral}
	\label{fig:scan_chain_sim}
\end{figure}\selectlanguage{italian}%



\selectlanguage{italian}%

\section{Sintesi su board FPGA}

Fare riferimento a \ref{board disp}.\selectlanguage{italian}%

\selectlanguage{italian}%



\chapter{Addizionatore a 7 operandi}

\selectlanguage{american}%

\AddToShipoutPicture{\BackgroundPicture{logo.png}{0}}

\selectlanguage{italian}%

\section{Traccia}

Realizzare un dispositivo VHDL che implementa il protocollo UART (a
partire da quello diffuso dalla Digilent). Collegare internamente,
oppure tramite inter- faccia fisica esterna alla board stessa, ad
un\textquoteright altra board oppure ad un PC previo utilizzo di un
physical RS232, due interfacce per trasmette e ricevere ottetti. Svolgere
l\textquoteright esercizio riutilizzando il VHDL messo a disposizione
da Digilent (e disponbile nel materiale del corso) commentando eventuali
ristrutturazioni del codice. (Opzionale) Sviluppare un\textquoteright architettura
per l\textquoteright implementazione del protocollo UART secondo il
paradigma PO/PC ex-novo, evidenziando le similarit`a/dissimilarit`a
con il progetto Digilent.\selectlanguage{italian}%



\selectlanguage{italian}%

\section{Soluzione}


\subsection{Schematico}

\subsubsection{GPIO}

\begin{figure}[H]
	\centering
	\includegraphics[scale=0.7]{esercizio16/images/GPIO.png}
	\caption{GPIO Schematic}
	\label{fig:GPIO_schematics}
\end{figure}

\subsection{Codice}

Progetto ISE: \href{run:progetti/GPIO/GPIO.xise}{GPIO ISE}


\subsubsection{Pad}

Questo componente, rappresentato in Figura \ref{fig:GPIO_schematics}
decide se l'operazione da effettuare � di scrittura o di lettura in
base al valore del segnale di enable. Quando enable � alto, si effettua
una scrittura e il segnale di input/output \textit{in\_out }viene
caricato con il valore del segnale di input, poi trasferito al segnale
di output. Quando enable � basso si effettua un'operazione di lettura
quindi, per evitare di disturbare il valore da leggere, viene spento
il buffer, ponendo il segnale in\_out ad alta impedenza ('Z'). Questo
comportamento � modellato tramite un costrutto dataflow \textit{with-select.}\\
Il componente in questione � osservabile a questo link: \href{run:progetti/GPIO/pad.vhd}{Pad}

\subsubsection{GPIO}

Questa � una top-level entity che si preoccupa di creare, tramite
un \textit{for-generate, }una serie di pad, il cui numero � settato
tramite un generic. \\
Il componente in questione � osservabile a questo link: \href{run:progetti/GPIO/gpio.vhd}{GPIO}\selectlanguage{italian}%



\selectlanguage{italian}%

\section{Simulazione}

In Figura \ref{fig:scan_chain_sim} � osservabile il funzionamento
del sistema appena descritto, utilizzando il testbench presente al
seguente link: \href{run:progetti/Boundary_Scan_Chain/boundary_scan_chain_testbench.vhd}{Boundary\_Scan\_Chain\_Testbench}.
\\
Come si nota � stata abilitata la modalit� shift register ponendo
ad 1 sia \textit{en}, che \textit{scan\_en}, in modo da shiftare il
valore 1 posto in ingresso a \textit{scan\_in}, che � visibile all'uscita
scan\_out esattamente dopo quattro colpi di clock, come c'era d'aspettarsi
dato che il numero di flip flop utilizzati � proprio quattro. Dopodich�
si � scelto di abilitare la modalit� registro e quindi a 55 ns sono
stati abbassati sia \textit{en} che \textit{scan\_en} e difatti all'uscita
\textit{dout} dei flip flop si osserva il valore 1010 posto in ingresso
precedentemente a \textit{din}.

\begin{figure}[H]
	\centering
	\includegraphics[scale=0.8]{esercizio06/images/scan_chain_simulazione.png}
	\caption{Simulazione della Boundary Scan Chain Behavioral}
	\label{fig:scan_chain_sim}
\end{figure}\selectlanguage{italian}%



\selectlanguage{italian}%

\section{Sintesi su board FPGA}

Fare riferimento a \ref{board disp}.\selectlanguage{italian}%

\selectlanguage{italian}%



\chapter{Moltiplicatori}

\selectlanguage{american}%

\AddToShipoutPicture{\BackgroundPicture{logo.png}{0}}

\selectlanguage{italian}%

\section{Traccia}

Realizzare un dispositivo VHDL che implementa il protocollo UART (a
partire da quello diffuso dalla Digilent). Collegare internamente,
oppure tramite inter- faccia fisica esterna alla board stessa, ad
un\textquoteright altra board oppure ad un PC previo utilizzo di un
physical RS232, due interfacce per trasmette e ricevere ottetti. Svolgere
l\textquoteright esercizio riutilizzando il VHDL messo a disposizione
da Digilent (e disponbile nel materiale del corso) commentando eventuali
ristrutturazioni del codice. (Opzionale) Sviluppare un\textquoteright architettura
per l\textquoteright implementazione del protocollo UART secondo il
paradigma PO/PC ex-novo, evidenziando le similarit`a/dissimilarit`a
con il progetto Digilent.\selectlanguage{italian}%



\selectlanguage{italian}%

\section{Soluzione}


\subsection{Schematico}

\subsubsection{GPIO}

\begin{figure}[H]
	\centering
	\includegraphics[scale=0.7]{esercizio16/images/GPIO.png}
	\caption{GPIO Schematic}
	\label{fig:GPIO_schematics}
\end{figure}

\subsection{Codice}

Progetto ISE: \href{run:progetti/GPIO/GPIO.xise}{GPIO ISE}


\subsubsection{Pad}

Questo componente, rappresentato in Figura \ref{fig:GPIO_schematics}
decide se l'operazione da effettuare � di scrittura o di lettura in
base al valore del segnale di enable. Quando enable � alto, si effettua
una scrittura e il segnale di input/output \textit{in\_out }viene
caricato con il valore del segnale di input, poi trasferito al segnale
di output. Quando enable � basso si effettua un'operazione di lettura
quindi, per evitare di disturbare il valore da leggere, viene spento
il buffer, ponendo il segnale in\_out ad alta impedenza ('Z'). Questo
comportamento � modellato tramite un costrutto dataflow \textit{with-select.}\\
Il componente in questione � osservabile a questo link: \href{run:progetti/GPIO/pad.vhd}{Pad}

\subsubsection{GPIO}

Questa � una top-level entity che si preoccupa di creare, tramite
un \textit{for-generate, }una serie di pad, il cui numero � settato
tramite un generic. \\
Il componente in questione � osservabile a questo link: \href{run:progetti/GPIO/gpio.vhd}{GPIO}\selectlanguage{italian}%



\selectlanguage{italian}%

\section{Simulazione}

In Figura \ref{fig:scan_chain_sim} � osservabile il funzionamento
del sistema appena descritto, utilizzando il testbench presente al
seguente link: \href{run:progetti/Boundary_Scan_Chain/boundary_scan_chain_testbench.vhd}{Boundary\_Scan\_Chain\_Testbench}.
\\
Come si nota � stata abilitata la modalit� shift register ponendo
ad 1 sia \textit{en}, che \textit{scan\_en}, in modo da shiftare il
valore 1 posto in ingresso a \textit{scan\_in}, che � visibile all'uscita
scan\_out esattamente dopo quattro colpi di clock, come c'era d'aspettarsi
dato che il numero di flip flop utilizzati � proprio quattro. Dopodich�
si � scelto di abilitare la modalit� registro e quindi a 55 ns sono
stati abbassati sia \textit{en} che \textit{scan\_en} e difatti all'uscita
\textit{dout} dei flip flop si osserva il valore 1010 posto in ingresso
precedentemente a \textit{din}.

\begin{figure}[H]
	\centering
	\includegraphics[scale=0.8]{esercizio06/images/scan_chain_simulazione.png}
	\caption{Simulazione della Boundary Scan Chain Behavioral}
	\label{fig:scan_chain_sim}
\end{figure}\selectlanguage{italian}%



\selectlanguage{italian}%

\section{Sintesi su board FPGA}

Fare riferimento a \ref{board disp}.\selectlanguage{italian}%

\selectlanguage{italian}%



\chapter{Divisori}

\selectlanguage{american}%

\AddToShipoutPicture{\BackgroundPicture{logo.png}{0}}

\selectlanguage{italian}%

\section{Traccia}

Realizzare un dispositivo VHDL che implementa il protocollo UART (a
partire da quello diffuso dalla Digilent). Collegare internamente,
oppure tramite inter- faccia fisica esterna alla board stessa, ad
un\textquoteright altra board oppure ad un PC previo utilizzo di un
physical RS232, due interfacce per trasmette e ricevere ottetti. Svolgere
l\textquoteright esercizio riutilizzando il VHDL messo a disposizione
da Digilent (e disponbile nel materiale del corso) commentando eventuali
ristrutturazioni del codice. (Opzionale) Sviluppare un\textquoteright architettura
per l\textquoteright implementazione del protocollo UART secondo il
paradigma PO/PC ex-novo, evidenziando le similarit`a/dissimilarit`a
con il progetto Digilent.\selectlanguage{italian}%



\selectlanguage{italian}%

\section{Soluzione}


\subsection{Schematico}

\subsubsection{GPIO}

\begin{figure}[H]
	\centering
	\includegraphics[scale=0.7]{esercizio16/images/GPIO.png}
	\caption{GPIO Schematic}
	\label{fig:GPIO_schematics}
\end{figure}

\subsection{Codice}

Progetto ISE: \href{run:progetti/GPIO/GPIO.xise}{GPIO ISE}


\subsubsection{Pad}

Questo componente, rappresentato in Figura \ref{fig:GPIO_schematics}
decide se l'operazione da effettuare � di scrittura o di lettura in
base al valore del segnale di enable. Quando enable � alto, si effettua
una scrittura e il segnale di input/output \textit{in\_out }viene
caricato con il valore del segnale di input, poi trasferito al segnale
di output. Quando enable � basso si effettua un'operazione di lettura
quindi, per evitare di disturbare il valore da leggere, viene spento
il buffer, ponendo il segnale in\_out ad alta impedenza ('Z'). Questo
comportamento � modellato tramite un costrutto dataflow \textit{with-select.}\\
Il componente in questione � osservabile a questo link: \href{run:progetti/GPIO/pad.vhd}{Pad}

\subsubsection{GPIO}

Questa � una top-level entity che si preoccupa di creare, tramite
un \textit{for-generate, }una serie di pad, il cui numero � settato
tramite un generic. \\
Il componente in questione � osservabile a questo link: \href{run:progetti/GPIO/gpio.vhd}{GPIO}\selectlanguage{italian}%



\selectlanguage{italian}%

\section{Simulazione}

In Figura \ref{fig:scan_chain_sim} � osservabile il funzionamento
del sistema appena descritto, utilizzando il testbench presente al
seguente link: \href{run:progetti/Boundary_Scan_Chain/boundary_scan_chain_testbench.vhd}{Boundary\_Scan\_Chain\_Testbench}.
\\
Come si nota � stata abilitata la modalit� shift register ponendo
ad 1 sia \textit{en}, che \textit{scan\_en}, in modo da shiftare il
valore 1 posto in ingresso a \textit{scan\_in}, che � visibile all'uscita
scan\_out esattamente dopo quattro colpi di clock, come c'era d'aspettarsi
dato che il numero di flip flop utilizzati � proprio quattro. Dopodich�
si � scelto di abilitare la modalit� registro e quindi a 55 ns sono
stati abbassati sia \textit{en} che \textit{scan\_en} e difatti all'uscita
\textit{dout} dei flip flop si osserva il valore 1010 posto in ingresso
precedentemente a \textit{din}.

\begin{figure}[H]
	\centering
	\includegraphics[scale=0.8]{esercizio06/images/scan_chain_simulazione.png}
	\caption{Simulazione della Boundary Scan Chain Behavioral}
	\label{fig:scan_chain_sim}
\end{figure}\selectlanguage{italian}%



\selectlanguage{italian}%

\section{Sintesi su board FPGA}

Fare riferimento a \ref{board disp}.\selectlanguage{italian}%

\selectlanguage{italian}%



\chapter{UART}

\selectlanguage{american}%

\AddToShipoutPicture{\BackgroundPicture{logo.png}{0}}

\selectlanguage{italian}%

\section{Traccia}

Realizzare un dispositivo VHDL che implementa il protocollo UART (a
partire da quello diffuso dalla Digilent). Collegare internamente,
oppure tramite inter- faccia fisica esterna alla board stessa, ad
un\textquoteright altra board oppure ad un PC previo utilizzo di un
physical RS232, due interfacce per trasmette e ricevere ottetti. Svolgere
l\textquoteright esercizio riutilizzando il VHDL messo a disposizione
da Digilent (e disponbile nel materiale del corso) commentando eventuali
ristrutturazioni del codice. (Opzionale) Sviluppare un\textquoteright architettura
per l\textquoteright implementazione del protocollo UART secondo il
paradigma PO/PC ex-novo, evidenziando le similarit`a/dissimilarit`a
con il progetto Digilent.\selectlanguage{italian}%



\selectlanguage{italian}%

\section{Soluzione}


\subsection{Schematico}

\subsubsection{GPIO}

\begin{figure}[H]
	\centering
	\includegraphics[scale=0.7]{esercizio16/images/GPIO.png}
	\caption{GPIO Schematic}
	\label{fig:GPIO_schematics}
\end{figure}

\subsection{Codice}

Progetto ISE: \href{run:progetti/GPIO/GPIO.xise}{GPIO ISE}


\subsubsection{Pad}

Questo componente, rappresentato in Figura \ref{fig:GPIO_schematics}
decide se l'operazione da effettuare � di scrittura o di lettura in
base al valore del segnale di enable. Quando enable � alto, si effettua
una scrittura e il segnale di input/output \textit{in\_out }viene
caricato con il valore del segnale di input, poi trasferito al segnale
di output. Quando enable � basso si effettua un'operazione di lettura
quindi, per evitare di disturbare il valore da leggere, viene spento
il buffer, ponendo il segnale in\_out ad alta impedenza ('Z'). Questo
comportamento � modellato tramite un costrutto dataflow \textit{with-select.}\\
Il componente in questione � osservabile a questo link: \href{run:progetti/GPIO/pad.vhd}{Pad}

\subsubsection{GPIO}

Questa � una top-level entity che si preoccupa di creare, tramite
un \textit{for-generate, }una serie di pad, il cui numero � settato
tramite un generic. \\
Il componente in questione � osservabile a questo link: \href{run:progetti/GPIO/gpio.vhd}{GPIO}\selectlanguage{italian}%



\selectlanguage{italian}%

\section{Simulazione}

In Figura \ref{fig:scan_chain_sim} � osservabile il funzionamento
del sistema appena descritto, utilizzando il testbench presente al
seguente link: \href{run:progetti/Boundary_Scan_Chain/boundary_scan_chain_testbench.vhd}{Boundary\_Scan\_Chain\_Testbench}.
\\
Come si nota � stata abilitata la modalit� shift register ponendo
ad 1 sia \textit{en}, che \textit{scan\_en}, in modo da shiftare il
valore 1 posto in ingresso a \textit{scan\_in}, che � visibile all'uscita
scan\_out esattamente dopo quattro colpi di clock, come c'era d'aspettarsi
dato che il numero di flip flop utilizzati � proprio quattro. Dopodich�
si � scelto di abilitare la modalit� registro e quindi a 55 ns sono
stati abbassati sia \textit{en} che \textit{scan\_en} e difatti all'uscita
\textit{dout} dei flip flop si osserva il valore 1010 posto in ingresso
precedentemente a \textit{din}.

\begin{figure}[H]
	\centering
	\includegraphics[scale=0.8]{esercizio06/images/scan_chain_simulazione.png}
	\caption{Simulazione della Boundary Scan Chain Behavioral}
	\label{fig:scan_chain_sim}
\end{figure}\selectlanguage{italian}%



\selectlanguage{italian}%

\section{Sintesi su board FPGA}

Fare riferimento a \ref{board disp}.\selectlanguage{italian}%

\selectlanguage{italian}%



\chapter{GPIO}

\selectlanguage{american}%

\AddToShipoutPicture{\BackgroundPicture{logo.png}{0}}

\selectlanguage{italian}%

\section{Traccia}

Realizzare un dispositivo VHDL che implementa il protocollo UART (a
partire da quello diffuso dalla Digilent). Collegare internamente,
oppure tramite inter- faccia fisica esterna alla board stessa, ad
un\textquoteright altra board oppure ad un PC previo utilizzo di un
physical RS232, due interfacce per trasmette e ricevere ottetti. Svolgere
l\textquoteright esercizio riutilizzando il VHDL messo a disposizione
da Digilent (e disponbile nel materiale del corso) commentando eventuali
ristrutturazioni del codice. (Opzionale) Sviluppare un\textquoteright architettura
per l\textquoteright implementazione del protocollo UART secondo il
paradigma PO/PC ex-novo, evidenziando le similarit`a/dissimilarit`a
con il progetto Digilent.\selectlanguage{italian}%



\selectlanguage{italian}%

\section{Soluzione}


\subsection{Schematico}

\subsubsection{GPIO}

\begin{figure}[H]
	\centering
	\includegraphics[scale=0.7]{esercizio16/images/GPIO.png}
	\caption{GPIO Schematic}
	\label{fig:GPIO_schematics}
\end{figure}

\subsection{Codice}

Progetto ISE: \href{run:progetti/GPIO/GPIO.xise}{GPIO ISE}


\subsubsection{Pad}

Questo componente, rappresentato in Figura \ref{fig:GPIO_schematics}
decide se l'operazione da effettuare � di scrittura o di lettura in
base al valore del segnale di enable. Quando enable � alto, si effettua
una scrittura e il segnale di input/output \textit{in\_out }viene
caricato con il valore del segnale di input, poi trasferito al segnale
di output. Quando enable � basso si effettua un'operazione di lettura
quindi, per evitare di disturbare il valore da leggere, viene spento
il buffer, ponendo il segnale in\_out ad alta impedenza ('Z'). Questo
comportamento � modellato tramite un costrutto dataflow \textit{with-select.}\\
Il componente in questione � osservabile a questo link: \href{run:progetti/GPIO/pad.vhd}{Pad}

\subsubsection{GPIO}

Questa � una top-level entity che si preoccupa di creare, tramite
un \textit{for-generate, }una serie di pad, il cui numero � settato
tramite un generic. \\
Il componente in questione � osservabile a questo link: \href{run:progetti/GPIO/gpio.vhd}{GPIO}\selectlanguage{italian}%



\selectlanguage{italian}%

\section{Simulazione}

In Figura \ref{fig:scan_chain_sim} � osservabile il funzionamento
del sistema appena descritto, utilizzando il testbench presente al
seguente link: \href{run:progetti/Boundary_Scan_Chain/boundary_scan_chain_testbench.vhd}{Boundary\_Scan\_Chain\_Testbench}.
\\
Come si nota � stata abilitata la modalit� shift register ponendo
ad 1 sia \textit{en}, che \textit{scan\_en}, in modo da shiftare il
valore 1 posto in ingresso a \textit{scan\_in}, che � visibile all'uscita
scan\_out esattamente dopo quattro colpi di clock, come c'era d'aspettarsi
dato che il numero di flip flop utilizzati � proprio quattro. Dopodich�
si � scelto di abilitare la modalit� registro e quindi a 55 ns sono
stati abbassati sia \textit{en} che \textit{scan\_en} e difatti all'uscita
\textit{dout} dei flip flop si osserva il valore 1010 posto in ingresso
precedentemente a \textit{din}.

\begin{figure}[H]
	\centering
	\includegraphics[scale=0.8]{esercizio06/images/scan_chain_simulazione.png}
	\caption{Simulazione della Boundary Scan Chain Behavioral}
	\label{fig:scan_chain_sim}
\end{figure}\selectlanguage{italian}%



\selectlanguage{italian}%

\section{Sintesi su board FPGA}

Fare riferimento a \ref{board disp}.\selectlanguage{italian}%

\selectlanguage{italian}%



\chapter{Firma digitale}

\selectlanguage{american}%

\AddToShipoutPicture{\BackgroundPicture{logo.png}{0}}

\selectlanguage{italian}%

\section{Traccia}

Realizzare un dispositivo VHDL che implementa il protocollo UART (a
partire da quello diffuso dalla Digilent). Collegare internamente,
oppure tramite inter- faccia fisica esterna alla board stessa, ad
un\textquoteright altra board oppure ad un PC previo utilizzo di un
physical RS232, due interfacce per trasmette e ricevere ottetti. Svolgere
l\textquoteright esercizio riutilizzando il VHDL messo a disposizione
da Digilent (e disponbile nel materiale del corso) commentando eventuali
ristrutturazioni del codice. (Opzionale) Sviluppare un\textquoteright architettura
per l\textquoteright implementazione del protocollo UART secondo il
paradigma PO/PC ex-novo, evidenziando le similarit`a/dissimilarit`a
con il progetto Digilent.\selectlanguage{italian}%



\selectlanguage{italian}%

\section{Soluzione}


\subsection{Schematico}

\subsubsection{GPIO}

\begin{figure}[H]
	\centering
	\includegraphics[scale=0.7]{esercizio16/images/GPIO.png}
	\caption{GPIO Schematic}
	\label{fig:GPIO_schematics}
\end{figure}

\subsection{Codice}

Progetto ISE: \href{run:progetti/GPIO/GPIO.xise}{GPIO ISE}


\subsubsection{Pad}

Questo componente, rappresentato in Figura \ref{fig:GPIO_schematics}
decide se l'operazione da effettuare � di scrittura o di lettura in
base al valore del segnale di enable. Quando enable � alto, si effettua
una scrittura e il segnale di input/output \textit{in\_out }viene
caricato con il valore del segnale di input, poi trasferito al segnale
di output. Quando enable � basso si effettua un'operazione di lettura
quindi, per evitare di disturbare il valore da leggere, viene spento
il buffer, ponendo il segnale in\_out ad alta impedenza ('Z'). Questo
comportamento � modellato tramite un costrutto dataflow \textit{with-select.}\\
Il componente in questione � osservabile a questo link: \href{run:progetti/GPIO/pad.vhd}{Pad}

\subsubsection{GPIO}

Questa � una top-level entity che si preoccupa di creare, tramite
un \textit{for-generate, }una serie di pad, il cui numero � settato
tramite un generic. \\
Il componente in questione � osservabile a questo link: \href{run:progetti/GPIO/gpio.vhd}{GPIO}\selectlanguage{italian}%



\selectlanguage{italian}%

\section{Simulazione}

In Figura \ref{fig:scan_chain_sim} � osservabile il funzionamento
del sistema appena descritto, utilizzando il testbench presente al
seguente link: \href{run:progetti/Boundary_Scan_Chain/boundary_scan_chain_testbench.vhd}{Boundary\_Scan\_Chain\_Testbench}.
\\
Come si nota � stata abilitata la modalit� shift register ponendo
ad 1 sia \textit{en}, che \textit{scan\_en}, in modo da shiftare il
valore 1 posto in ingresso a \textit{scan\_in}, che � visibile all'uscita
scan\_out esattamente dopo quattro colpi di clock, come c'era d'aspettarsi
dato che il numero di flip flop utilizzati � proprio quattro. Dopodich�
si � scelto di abilitare la modalit� registro e quindi a 55 ns sono
stati abbassati sia \textit{en} che \textit{scan\_en} e difatti all'uscita
\textit{dout} dei flip flop si osserva il valore 1010 posto in ingresso
precedentemente a \textit{din}.

\begin{figure}[H]
	\centering
	\includegraphics[scale=0.8]{esercizio06/images/scan_chain_simulazione.png}
	\caption{Simulazione della Boundary Scan Chain Behavioral}
	\label{fig:scan_chain_sim}
\end{figure}\selectlanguage{italian}%



\selectlanguage{italian}%

\section{Sintesi su board FPGA}

Fare riferimento a \ref{board disp}.\selectlanguage{italian}%

\selectlanguage{italian}%



\selectlanguage{american}%
\end{mainmatter}\selectlanguage{italian}%

\end{document}
