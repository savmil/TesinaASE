\selectlanguage{italian}%

\section{Sintesi su board FPGA\label{tester dispositivi seriali}}

\lstinputlisting [language=VHDL,caption={Definizione del componente che gestisce il caricamento dei dati},firstline=32] {/progetti/riconoscitore_stringa/tester_dispositivi.vhd}

Il seguente codice in VHDL descrive un tester utilizzato, con piccole
differenze dove occorre, utilizzato per tutti i dispositivi non solo
combinatoriali presenti in tale elaborato, difatti al suo interno
� presente un process che genera un segnale di start attivo fino a
quando il contatore al di sopra di esso non termina, questo perch�
tale segnale non deve essere attivo continuamente altrimenti i dispositivi
ricomincerebbero a computare non appena tornerebbero nello stato di
riposo, ma deve essere attivo per un tempo tale affinch� venga rilevato
dal dispositivo che deve essere avviato.

Vengono utilizzati gli switch per inserire sia la stringa da riconoscere,
oltre a quella con cui si deve effettuare il confronto, tali valori
sono sorretti da registri;

il led zero si attiva se il riconoscimento � andato a buon fine, il
led uno nel caso contrario;

i led sei e sette occorrono a determinare: quali registri per contenere
i dati si sono selezionati (se acceso il led sei il dato da voler
confrontare, se acceso il sette la stringa da riconoscere), se si
� avviata la computazione (entrambi i led accesi) o nel caso in cui
sono spenti entrambi venga effettuato il reset del dispositivo per
effettuare una successiva computazione;

il reset degli altri componenti invece � abilitato della pressione
del pulsante tre, invece il pulsante zero permette la selezione dei
registri e l' avvio della macchina a stati finiti;

per caricare i dati gli switch devono essere impostati prima che il
led relativo al registro venga acceso, vale anche per gli altri tester
utilizzati.\selectlanguage{italian}%

